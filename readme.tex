\documentclass[12pt]{article}
\usepackage{etex}
\usepackage{amsmath,amsxtra,amssymb,latexsym,amscd,amsthm}
\usepackage{indentfirst}
\usepackage{epigraph}
\usepackage{tabvar}
\usepackage{ifpdf}
\usepackage[english]{babel}
\usepackage[mathscr]{eucal}
\usepackage{graphics, graphicx}
\usepackage{pstricks,pst-node,pst-tree}
\usepackage{caption}
\usepackage{subcaption}
\usepackage{fancybox}
\usepackage[colorlinks=true]{hyperref}  
\usepackage[a4paper,left=15mm,right=10mm,top=15mm,bottom=15mm]{geometry}
\renewcommand{\baselinestretch}{1.5}

\title{RedBlackTreeViewer Instruction}
\author{CS 150: Data Structures and Algorithm}
\date{}

\begin{document}
\maketitle

\section{Getting started}
Check that you have the following Java files in the current directory:
\begin{itemize}
	\item \texttt{BinaryTreePanel.java}
	\item \texttt{Node.java}
	\item \texttt{RBTree.java}
	\item \texttt{RedBlackTreeViewer.java}
\end{itemize}

Compile all the above files by typing in the terminal
$$\texttt{javac *.java}$$

To run the program, type in the terminal
$$\texttt{java RedBlackTreeViewer}$$

A user interface will pop up, allowing you to interact with a red black tree.

\section{Using the program}
The user interface consists of a text box and $4$ buttons:
\begin{center}
	\includegraphics[scale = 0.5]{UI.png}

	Figure 1: The user interface.
\end{center}

The program works as follows:
\begin{itemize}
	\item You can enter one {\it integer} to the text box and press {\it Add} to add it to the current tree, which is initially empty.
	\item You can enter an array of {\it integers} to the text box and press {\it Add All} to add all of them one by one to the tree. Use \texttt{space} as the separator between integers. For example, $$\texttt{-2 -10 4 56 932}$$
	\item You can enter one {\it integer} to the text box and press {\it Remove} to remove it from the current tree.
	\item You can press {\it Remove All} to delete all nodes in the tree. Note that this operation does not take any input. What you enter to the text box has no impact on it.
\end{itemize}

\begin{center}
	\includegraphics[scale = 0.5]{screenshot.png}

	Figure 2: A screenshot of the program.
\end{center}

The tree will then be modified and displayed accordingly so that you can check with your work. The source code in \texttt{RBTree.java} might also be useful for reference.

{\bf CAUTION.} This program is not made to handle invalid inputs, including but not limited to:
\begin{itemize}
	\item Input is not integer (double, string, ...).
	\item The separator is not \texttt{space}. For example, $\text{1,2,3,4,5}$ instead of 1 2 3 4 5.
	\item A duplicate value is entered.
	\item {\it Add} or {\it Remove} is pressed when the text box has more than one integer.
\end{itemize}

When an invalid input is entered, the program crashes so horribly that you will wish you could unsee that bloody scene. In the worst case that such a tragedy does happen, take a deep breath, reflect on yourself, close the existing interface, then re-run the program. And don't mess up again.

If you encounter any problem, contact Huy Nguyen at \texttt{nguyenha@lafayette.edu}.

\end{document}